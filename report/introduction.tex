\section{Introduction}
\label{sec:intro}
Artificial neural networks (ANNs) are machine learning models loosely inspired by biological neural networks found in human and animal brains. ANNs consist of a set of nodes, called artificial neurons, connected to each other by edges. Nodes can send and receive signals, i.e. real numbers, through their edges. Upon receiving a signal from other nodes, a node processes said signal and can signal other nodes, i.e, a node's output is a function of all its inputs. This project explores DNNs of two kinds: and CNNs.

\textbf{DNNs.} DNNs, i.e., Deep Neural Networks, have nodes aggregated into layers. Usually nodes from the first layer forward signals to nodes in the next layer, which, in turn, forward signals to the next layer and the forwarding is repeated until the last layer is reached.

\textbf{CNNs.} CNNs, i.e., convolutional neural networks, are a subclass of DNNs. CNNs are most commonly used in computer vision.
