\section{Reproducing the experiments}
\label{sec:repro}
The code to reproduce the both experiments is available at \href{https://github.com/GabrieleMaurina/statistical\_methods\_for\_machine\_learning\_project}{github.com/GabrieleMaurina/statistical\_methods\_for\_machine\_learning\_project}. Note that the experiments can take up to several hours to complete, depending on the hardware used. To reproduce the experiment, follow the steps:
\begin{itemize}
    \item clone the \href{https://github.com/GabrieleMaurina/statistical\_methods\_for\_machine\_learning\_project}{repository};
    \item install \href{https://pypi.org/project/tensorflow/}{Tensorflow}, \href{https://pypi.org/project/numpy/}{Numpy} and \href{https://pypi.org/project/matplotlib/}{Matplotlib};
    \item download the \href{https://www.kaggle.com/moltean/fruits}{image dataset};
    \item unzip the dataset;
    \item move the folder fruits-360 to the root of the cloned repository;
    \item run experiment 1 with \texttt{python experiment\_1.py};
    \item manually overwrite the \texttt{results\_exp\_1} folder with the newly created \texttt{results} folder;
    \item run experiment 2 with \texttt{python experiment\_2.py};
    \item manually overwrite the \texttt{results\_exp\_2} folder with the newly created \texttt{results} folder;
    \item cd into \texttt{plots/};
    \item generate plots with \texttt{python plots.py};
    \item run \texttt{stats.py} to get some stats.
\end{itemize}
Both experiment 1 and experiment 2 save the results into the \texttt{results} folder. It is important to avoid mixing the results and to move them to the corresponding folders, as indicated in the steps above. This manual step is necessary because the folders \texttt{results\_exp\_1} and \texttt{results\_exp\_2} are versioned and it is undesirable to accidentally overwrite them when testing the experiment scripts. Once one is satisfied with the results, then the folder \texttt{results} should be renamed \texttt{results\_exp\_<number>}. Furthermore, the \texttt{plots.py} script takes the results from the folders \texttt{results\_exp\_1} and \texttt{results\_exp\_2}.

\textbf{Demo.} The repository also contains a demonstration of a model predicting images. To run the demonstration first run the script \texttt{demo\_model.py}, which will create, train and save a model. Then run \texttt{demo.py}, which will load the model, iteratively load images from the testing dataset and for each image, show it on screen, make a prediction and show the true label.
